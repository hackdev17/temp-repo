\documentclass[12pt]{article}

% packages
\usepackage[a4paper,lmargin=1.25in,rmargin=1in,tmargin=1in,bmargin=1in]{geometry}
\usepackage{times}
\usepackage{amsmath}
\usepackage{graphicx}
\usepackage{multirow}
\usepackage{enumitem}
\usepackage{tikz}
\usepackage{calc}
\usepackage{hyperref}

% detailes
\title{SOCIAL CONNECT RESPONSIBILITY REPORT}
\author{Shreyas}
\date{February 15, 2024}

% presets
\usetikzlibrary{shapes}
\linespread{1.5}
\pagestyle{empty}
\hypersetup{
	colorlinks,
	urlcolor={blue!80!black},
	linkcolor=black
}
\urlstyle{same}

% commands
\newcommand{\chapter}[1]{
	\section*{#1}
	\addcontentsline{toc}{section}{#1}
}

\newcommand{\img}[2]{\scalebox{#1}{\includegraphics{#2}}}

\newcommand{\border}{
	\tikz[overlay, remember picture]{
		\draw ([shift={(.5in,-.5in)}]current page.north west)
		rectangle ++ (\paperwidth-1in,-\paperheight+1in);
	}
}

\newcommand{\nborder}{
	\tikz[overlay, remember picture]{
		\draw ([shift={(.5in,-.5in)}]current page.north west)
		rectangle ++ (\paperwidth-1in,-\paperheight+1in)
		node[below left,yshift=-.5em]{
			~~Social Connect and Responsibility (BSCK307) \hspace{22em} \thepage~~
		};
	}
}

\newcommand{\rborder}{
	\tikz[overlay, remember picture]{
		\draw ([shift={(.5in,-.5in)}]current page.north west)
		rectangle ++ (\paperwidth-1in,-\paperheight+1in)
		node[below left,yshift=-.5em]{
			~~Social Connect and Responsibility (BSCK307) \hspace{21em} (\thepage)~~
		};
	}
}

\begin{document}
	\pagenumbering{roman}
	\maketitle \thispagestyle{empty} \border \newpage

	\setcounter{page}{1}
	\rborder \vspace{-5em} \tableofcontents \thispagestyle{empty} \newpage

	\thispagestyle{empty} \section*{References}
	Mangifera indica \dotfill ~~\textbf{1} \\
	\indent Wikipedia : Mango\\
	\indent\indent \url{https://simple.wikipedia.org/wiki/Mango} \\
	Organic Farming \dotfill ~~\textbf{6} \\
	\indent Wikipedia : Organic farming\\
	\indent\indent \url{https://en.wikipedia.org/wiki/Organic_farming}

	\section*{List of Tables}
	\noindent Biomass production \& Nutrient content of green manure crops \dotfill ~~\textbf{9} \\
	Nutrient content of green manure crops \dotfill ~~\textbf{9} \\ ~ \rborder \newpage

	\pagenumbering{arabic}
	\setcounter{page}{1}
	\chapter{Module 1 $-$ PLANTATION AND ADOPTATION OF TREE}
	\subsubsection{Introduction}
	\par Mangifera indica, commonly known as mango, is a species of flowering plant in the family Anacardiaceae. It is a large fruit tree, capable of growing to a height of 30 metres (100 feet). There are two distinct genetic populations in modern mangoes – the "Indian type" and the "Southeast Asian type".

	\subsubsection{Fruit}
	It is irregular, egg-shaped fruit which is a fleshy drupe. Mangoes are typically 8–12 cm (3–5 in) long and greenish yellow in color. The fruits can be round, oval, heart, or kidney shaped.Mango fruits are green when they are unripe. The interior flesh is bright orange and soft with a large, flat pit in the middle.Mangoes are mature in April and May. The skin and pulp account for 85\% of the mango's weight, and the remaining 15\% comes from the seed

	\subsubsection{Harvesting}
	Red-yellow flowers appear at the end of winter, and also at the beginning of spring. Both male and female flowers are borne on same tree. Climatic conditions have a significant influence on the time of flowering. In South Asia, flowering starts in December in the south, in January in Bengal, in February in eastern Uttar Pradesh and Bihar, and in February–March in northern India. The duration of flowering is 20–25 days for the Dasheri variety, while panicle emergence occurs in early December and flower opening is completed by February. The Neelum variety produces two crops a year in Kanyakumari, Tamil Nadu, but it flowers only once in North Indian conditions.
	\nborder
	\begin{center}
		\vspace{1em}
		\img{.65}{Plant/mango.jpg} \\ Mango grown for 2 months
	\end{center} \newpage

	\chapter{Module 2 $-$ HERITAGE WALK AND CRAFTS CORNER}
	\subsection{PILIKULA VILLAGE NISARGADHAM} \nborder

	\begin{center}
		\img{.8}{Heritage/Plikula Village Nisargadham-2.jpg}
	\end{center}

	\par Dakshina Kannada district too is rich in heritage and culture. Spread across 350 acres, Pilikula Nisargadham offers an array of activities one can think of. Botanical Garden, Medicinal Garden, Guthu Mane, Pottery, Handloom,Blacksmith, nursery, and Folklore gallery are the few main attractions of the Pilikula Nisargadham. Dakshina Kannada’s economy primarily depends on tourism, agriculture, and its by-products. The region is bestowed with heritage, culture, and historical marvels. At Pilikula Nisargadham, after deep research on the village and cultural aspects, a heritage village is recreated. This heritage village facilitates the visitors on how the various activities like handloom, pottery making, black-smithy, stone carving, carpentry, and many other hand-crafted activities are performed. This gives visitors a first-hand experience of the various rural activities.

	\subsubsection{History}
	\par Established in the year 1990, Pilikula in Tulu means ‘Tiger Pond’ and is about five acres in area. It is the lake where tigers once upon a time used to come to quench their thirsts, hence the name. Tulunadu culture of Dakshina Kannada district is one of the exquisite cultures which is becoming extinct. As we walk in we will find a navigation board mentioning things to see in the artisan village-like beaten rice mill unit, oil extraction, stone carvings, black smithy, handloom, carpentry, cane, and bamboo handicrafts. The other arts and crafts include Pottery, Handloom, Wood Carving, Toddy Jaggery, Fish Net Weaving, and Jasmine Culture. Pilikula artisan village was established to protect and preserve the old age unique traditional art and crafts forms and depict them to the newer generation so that the tradition goes on. \\

	\par One of the traditional and hand-crafted processes of molding iron and steel to produce various domestic and agricultural equipment is Black Smithy. In this process, the metal is heated until it turns red and becomes soft enough, and is hammered to the desired shape.
	The most significant art of Dakshina Kannada is stone carving and sculpturing. Rich in natural rocks and stones, the district uses granite, soapstone, slate, and black stone for sculpting and carving the idols of deities and Nagas. \\

	\par \nborder Pilikula Nisargadham Village apart from the artisan village also consists of many other attractions like a boating center, a biological park, an arboretum, a science center, a water amusement park, and a golf course.

	\subsubsection{Notable places of interest $-$ Guthu Mane}
	\par An architectural marvel, Guthu Mane is a traditional house of the coastal landlords. As the landlords of the yester years are progressing or migrating to the city life, the legacy of traditional and cultural heritage homes is becoming extinct. Look for intricately carved interiors, wooden pillars, ancient furniture and artifacts, and brass and copper antiques and vessels which are noteworthy. A huge Guthu Mane is restored and reinstalled to educate the visitors about the tradition, heritage, and culture of the region. The house also exhibits the dance forms, traditional attire, and other forms of folk arts and folk sports like Kambala, Yakshagana, etc. \\

	\par The architecture of these homes followed Vastu principles. Guthu houses look like a mini temple from the outside. Characterised by sloping roofs with Mangalore tiles and huge courtyards that overlook paddy fields, Guthu houses are a reminder of a bygone era. They symbolised the wealth and prestige of the wealthy landlords who built these houses. Exquisite care was taken to ensure that not only did Guthu houses evoke grandeur but were also built keeping in mind Mangalore’s often oppressive heat and humidity as a coastal region. \newpage

	\par \nborder The open space in front of the house beyond the main gate, which was called jaal, would typically have a shelter made from woven dried coconut leaves that would serve as protection from the extreme heat. All Guthu houses had a veranda that acted as a sit-out and beautiful doors made from intricately carved teakwood or rosewood. Ornate and exquisitely carved wooden pillars adorned the interiors of the inner courtyard. These pillars upheld a decorated wooden ceiling, which usually had representations of mango and floral designs, foliage, and coin motifs. Solid wooden beams called jantis supported this grand ceiling.

	\begin{center}
		\img{.25}{Heritage/Interior-of-Guthu-Mane.jpg} ~ \img{.25}{Heritage/Ancient-Artefacts-at-Guthu-Mane-Pilikula.jpg} \vspace{.5em} \\ \img{.035}{Vishal_Akash.jpg} ~ \img{.03}{Shreyas_Sudhanva.jpg} \\
		The view inside Guttu Mane and Artifacts \\ ~ \\
		\img{.025}{traditional.jpg} ~ \img{.025}{kitchen.jpg} \\
		Traditions and Kitchen view
	\end{center} \newpage

	\par The souvenir shop or the outlet at Pilikula called ‘Parampara’ has some of the most interesting artifacts, handicrafts, hand-crafted vessels in terracotta, bamboo handicrafts, clothes, and many other products made at the artisanal village itself. You can also buy cold-pressed edible oils, hand-pounded beaten rice, organic jaggery, and much more. The intention is purely to promote traditional and cultural heritage.

	\begin{center}
		\img{.025}{Heritage/wooden_part.jpg} ~ \img{.105}{Heritage/wooden_part2.jpg} \\
		The wooden furniture
	\end{center}

	\par Guthu houses usually featured a lot of woodwork. Wooden swings where the head of the house sat, chests, cabinets, shelves, chairs, writing desks, reclining chairs and the quintessential vakil bench were all part of these homes. They all featured beautifully intricate carvings, which were sometimes inlaid with ivory. \nborder \newpage

	\chapter{Module 3 $-$ ORGANIC FARMING AND WASTE MANAGEMENT}

	\subsection{Organic Farming}
	\par Organic farming is a holistic approach to agriculture that prioritizes environmental sustainability, biodiversity conservation, and the health of ecosystems, consumers, and farmers alike. Unlike conventional farming, which relies heavily on synthetic pesticides, fertilizers, and genetically modified organisms (GMOs), organic farming emphasizes natural and traditional methods to cultivate crops and raise livestock. \\

	\noindent \textbf{Key Principles of Organic Farming}
	\begin{enumerate}[label=\textbf{\arabic*.}]
		\item \textbf{Soil Health} \\
		Organic farmers focus on nurturing soil health through practices such as crop rotation, composting, and the use of natural fertilizers like manure and cover crops. Healthy soil supports the growth of strong, nutrient-rich plants while minimizing the need for chemical inputs.

		\item \textbf{Biodiversity} \\
		Organic farming promotes biodiversity by creating diverse ecosystems within and around farms. This includes planting a variety of crops, preserving natural habitats, and avoiding the use of synthetic chemicals that can harm beneficial insects, birds, and other wildlife.

		\item \textbf{Chemical-Free Pest and Weed Management} \\
		Instead of relying on synthetic pesticides and herbicides, organic farmers employ techniques such as crop rotation, natural predators, mechanical cultivation, and mulching to manage pests and weeds. These methods reduce chemical exposure for farmers and consumers while maintaining ecological balance. \nborder \newpage

		\item \textbf{Animal Welfare} \\
		In organic livestock farming, animals are treated with care and respect. They have access to outdoor areas for grazing and exercise, and their diets are free from synthetic hormones and antibiotics. Organic standards also prohibit the use of genetically modified feed.

		\item \textbf{Sustainability} \\
		Organic farming aims to minimize environmental impact by conserving water, energy, and natural resources. Practices such as rainwater harvesting, agroforestry, and renewable energy adoption contribute to the long-term sustainability of agricultural systems.
	\end{enumerate}

	\noindent \textbf{Benefits of Organic Farming}
	\begin{enumerate}[label=\textbf{\arabic*.}]
		\item \textbf{Healthier Food} \\
		Organic produce is free from synthetic pesticides and GMOs, making it safer and healthier for consumers. Studies have shown that organic fruits and vegetables contain higher levels of vitamins, minerals, and antioxidants.

		\item \textbf{Environmental Protection} \\
		By reducing reliance on chemical inputs and promoting biodiversity, organic farming helps preserve soil fertility, water quality, and ecosystem health. It also mitigates greenhouse gas emissions and contributes to climate change adaptation.

		\item \textbf{Support for Rural Communities} \\
		Organic farming can provide economic opportunities for small-scale farmers and rural communities. By fostering local food systems and direct marketing channels, organic agriculture strengthens local economies and enhances food security.
	\end{enumerate}

	\par In conclusion, organic farming offers a sustainable alternative to conventional agriculture that prioritizes ecological integrity, human health, and social equity. By embracing organic practices and supporting organic producers, individuals can contribute to building a more resilient and regenerative food system for future generations. \nborder \newpage

	\subsubsection{Need}
	\begin{itemize}
		\item Increase in population make compulsion to agriculture Production , but to , increase it further , in sustainable manner.
		\item Natural balance needs to be maintained at all cost for existence of life and property.
		\item Agrochemicals which are product from fossil fuel and are not renewable and are diminishing in availability.
		\item It may also cost heavily on our foreign exchange in future.
	\end{itemize}

	\subsubsection{GREEN MANURE}
	\begin{center}
		\begin{tabular}{ccccc}
			\img{.55}{Waste/crotalaria_juncea.jpg} & \img{.25}{Waste/sesbania_rostrata.jpg} & \img{.2295}{Waste/cow_pea.jpg} & \img{.275}{Waste/cluster_bean.jpg} & \img{.1975}{Waste/sesbania_bispinosa.jpg} \\
			Crotalaria juncea & Sesbania rostrata & Cow pea & Cluster bean & Sesbania aculeata \\
		\end{tabular} \\ ~ \\
		{\large Green Manure Crops}
	\end{center}

	\begin{itemize}
		\item Green undecomposed material used as manure is called green manure.
		\item It is obtained in two ways :
		\begin{itemize}
			\item By growing green manure crops or by collecting green leaf (along with twigs) from plants grown in wastelands, field bunds and forest.
			\item Green manuring is growing in the field plants usually belonging to leguminous family and incorporating into the soil after sufficient growth.
		\end{itemize}
		\item The plants that are grown for green manure known as green manure crops.
		\item The most important green \textit{manurecrops} are \textit{sunnhemp}, \textit{dhaincha}, \textit{pillipesara}, clusterbeansand \textit{Sesbania rostrata}.
	\end{itemize} \nborder \newpage

	\begin{center}
		\begin{tabular}{|l|c|c|c|} \hline
			\multicolumn{1}{|c|}{Crop} & Age (Days) & Day Matter (t/ha) & N accumulated \\ \hline
			Sesbania rostrata & 60 & 23.2 & 133 \\
			Sunnhemp & 60 & 30.6 & 134 \\
			Cow pea & 60 & 23.2 & 134 \\
			Pillipesara & 60 & 25.0 & 102 \\
			Cluster bean & 50 & ~~3.2 & ~~91 \\
			Sesbania rostrata & 50 & ~~5.0 & ~~96 \\ \hline
		\end{tabular} \\ \vspace{.5em} Table : Biomass production and N accumulation of green manure crops

		\vspace{2em}
		\begin{tabular}{|l|l|c|c|c|} \hline
			\multicolumn{1}{|c|}{\multirow{2}{*}{Plant}} & \multicolumn{1}{c|}{\multirow{2}{*}{Scientific Name}} & \multicolumn{3}{c|}{Nutrient content on air dry basis} \\ \cline{3-5}
			 & & ~~~~~N~~~~~ & ~~~P2O5~~~ & K \\ \hline
			Sunhemp & Crotataria jucea & 2.30 & 0.50 & 1.80 \\
			Dhaincha & Sebania acuteata & 3.50 & 0.60 & 1.20 \\
			Sesbania & Sebania speciosa & 2.71 & 0.53 & 2.21 \\ \hline
		\end{tabular} \\ \vspace{.5em} Table : Nutrient content of green manure crops
	\end{center}

	\subsubsection{Basic Steps of Organic Farming}
	\begin{enumerate}[label=\textbf{\arabic*.}]
		\item Conversion of land from conventional management to organic management
		\item Management of the entire surrounding system to ensure biodiversity and sustainability of the system
		\item Crop production with the use of alternative sources of nutrients such as crop rotation, residue management, organic manures and biological inputs.
		\item Management of weeds and pests by better management practices, physical and cultural means and by biological control system
		\item Maintenance of live stock in tandem with organic concept and make them an integral part of the entire system
	\end{enumerate} \nborder \newpage

	\subsection{Dry Waste management}
	\par Dry waste refers to any waste that does not decompose or decay easily and typically does not contain moisture. This category of waste includes materials such as plastics, metals, glass, paper, cardboard, and textiles. Dry waste is often generated from households, businesses, industries, and institutions. It is distinct from wet waste, which includes organic materials like food scraps and yard waste. \\

	\par Managing dry waste is crucial for environmental sustainability as improper disposal can lead to pollution, habitat destruction, and resource depletion. Recycling, reusing, and reducing dry waste are key strategies to mitigate its impact. Recycling facilities sort and process dry waste materials to produce new products, conserving resources and reducing the need for raw materials extraction. Reusing items such as glass jars, metal containers, and cloth bags helps minimize the generation of new waste. Additionally, reducing consumption and opting for eco-friendly alternatives can help lessen the overall amount of dry waste generated. Proper waste management practices, including segregation, collection, transportation, and disposal, are essential to ensure the responsible handling of dry waste and its sustainable management. \nborder \newpage

	\subsubsection{Utilizing Waste Newspaper for Flower Pot Creation}
	\begin{center}
		\img{.35}{Waste/Table.jpg}
	\end{center}
	\textbf{Introduction} \\
	\par The global waste crisis poses a significant challenge, with newspapers being a notable contributor. However, innovative approaches can turn this waste into a resource. One such endeavor involves crafting flower pots from discarded newspapers. This report explores the process, benefits, and implications of creating flower pots using waste newspaper. \\

	\noindent \textbf{Process of Making Flower Pots}
	\begin{enumerate}[label=\textbf{\arabic*.}]
		\item \textbf{Material Collection} \\
		Gather discarded newspapers.
		\item \textbf{Preparation} \\
		Cut the newspapers into strips or use them as they are.
		\item \textbf{Molding} \\
		Wrap the strips around a mold (e.g, a bottle or a can) to create the desired pot shape.
		\item \textbf{Adhesion} \\
		Apply a natural adhesive like flour paste or starch to secure the layers.
		\nborder \newpage
		\item \textbf{Drying} \\
		Allow the pot to dry thoroughly, either naturally or using gentle heat.
	\end{enumerate}

	\begin{center}
		\img{.15}{Waste/Paper Craft-0} ~ \img{.15}{Waste/Paper Craft-1} ~ \img{.15}{Waste/Paper_Craft} \\ The Paper Flower Pot making
	\end{center}

	\noindent \textbf{Advantages of Newspaper Flower Pots}
	\begin{enumerate}[label=\textbf{\arabic*.}]
		\item \textbf{Environmentally Friendly} \\
		Recycling newspaper reduces landfill waste and minimizes the need for new materials.
		\item \textbf{Cost-effective} \\
		Newspaper is readily available and inexpensive, making it an affordable option for pot making.
		\item \textbf{Biodegradable} \\
		Unlike plastic pots, newspaper pots decompose naturally, reducing environmental impact.
		\item \textbf{Customizable} \\
		These pots can be crafted in various shapes and sizes to suit different plant types and aesthetics.
		\item \textbf{Educational Value} \\
		Making newspaper pots can be an educational activity for children, teaching them about recycling and sustainability.
	\end{enumerate} \nborder \newpage

	\noindent \textbf{Implications and Challenges}
	\begin{enumerate}[label=\textbf{\arabic*.}]
		\item \textbf{Durability} \\
		Newspaper pots may not be as durable as plastic ones, requiring careful handling.
		\item \textbf{Moisture Resistance} \\
		While newspaper can hold moisture, it may degrade faster when constantly exposed to water, affecting pot longevity.
		\item \textbf{Aesthetics} \\
		The appearance of newspaper pots may not appeal to all users, although they can be decorated or painted to enhance visual appeal.
		\item \textbf{Market Acceptance} \\
		Commercial viability may depend on consumer acceptance and willingness to adopt eco-friendly alternatives.
	\end{enumerate}

	\begin{center}
		\img{.065}{Waste/Gift.jpg} \\ \textbf{Paper Flower Pot give away}
	\end{center}

	\nborder \newpage

	\chapter{Module 4 $-$ WATER CONSERVATION}
	\subsection{Water Conservation}
	\par Water conservation refers to the responsible management and efficient use of water resources to ensure their sustainability for current and future generations. It involves reducing water wastage, promoting water-saving technologies, and adopting practices that help preserve freshwater supplies. \\

	\noindent \textbf{Several strategies can be implemented to conserve water}
	\begin{enumerate}[label=\textbf{\arabic*.}]
		\item \textbf{Efficient Water Use} \\
		Encouraging individuals, industries, and agriculture to use water more efficiently through technologies like low-flow fixtures, drip irrigation, and water-efficient appliances.

		\item \textbf{Education and Awareness} \\
		Raising awareness about the importance of water conservation through campaigns, educational programs, and community engagement initiatives.

		\item \textbf{Rainwater Harvesting} \\
		Collecting and storing rainwater for later use in activities such as irrigation, gardening, and flushing toilets.

		\item \textbf{Greywater Recycling} \\
		Treating and reusing greywater (wastewater from sinks, showers, and washing machines) for non-potable purposes like landscape irrigation.

		\item \textbf{Infrastructure Improvements} \\
		Investing in water infrastructure upgrades and repairs to minimize leaks and losses in distribution systems.
		\nborder \newpage
		\item \textbf{Policy and Regulation} \\
		Implementing water conservation policies, regulations, and incentives at the local, regional, and national levels to promote responsible water use and discourage wasteful practices.

		\item \textbf{Landscaping Practices} \\
		Adopting water-wise landscaping techniques such as xeriscaping, which uses drought-tolerant plants and efficient irrigation methods to minimize outdoor water usage.

		\item \textbf{Industry Best Practices} \\
		Encouraging industries to adopt water-saving technologies and practices in their operations, such as recycling process water and implementing water-efficient manufacturing processes.
	\end{enumerate}

	\par Water conservation is crucial for mitigating water scarcity, protecting ecosystems, and ensuring equitable access to clean water for all. By implementing effective water conservation measures, individuals, communities, and governments can contribute to the sustainable management of water resources and build a more resilient future in the face of growing water challenges. \nborder \newpage

	\subsection{Rain Water Harvesting}
	\par Water conservation in villages through rainwater harvesting is a vital strategy to address water scarcity and improve access to clean water in rural areas. Rainwater harvesting involves capturing and storing rainwater for later use, reducing reliance on unsustainable groundwater sources and enhancing water resilience in communities.

	\begin{center}
		\img{.935}{Water/harvest_bank.jpg} ~ \img{.5}{Water/harvest_roof.jpg}
	\end{center}

	\noindent \textbf{In villages, rainwater harvesting can be implemented through various methods}
	\begin{enumerate}[label=\textbf{\arabic*.}]
		\item \textbf{Rooftop Rainwater Harvesting} \\
		Collecting rainwater from rooftops using gutters and downspouts, which is then stored in tanks or reservoirs for domestic use such as drinking, cooking, and sanitation.

		\item \textbf{Surface Water Harvesting} \\
		Building structures such as check dams, contour trenches, and percolation ponds to capture rainwater runoff from hillsides or sloped terrain, replenishing groundwater and providing water for irrigation and livestock.

		\item \textbf{Community Rainwater Harvesting Systems} \\
		Constructing community-level rainwater harvesting structures like small dams, ponds, or tanks to collect rainwater for shared use among villagers, particularly during dry seasons or droughts.

		\item \textbf{Farm-level Rainwater Harvesting} \\
		Implementing techniques such as contour farming, micro-catchments, and farm ponds to capture and store rainwater for agricultural purposes, enhancing crop yields and livelihoods while conserving water.
		\nborder \newpage
		\item \textbf{Awareness and Capacity Building} \\
		Educating villagers about the benefits of rainwater harvesting and providing training on how to design, build, and maintain rainwater harvesting systems effectively.

		\item \textbf{Policy Support and Funding} \\
		Encouraging government support and funding for rainwater harvesting initiatives in rural areas through subsidies, grants, and incentives to promote widespread adoption and sustainability.
	\end{enumerate}

	\par By promoting rainwater harvesting in villages, communities can become more self-reliant in water management, reduce vulnerability to water scarcity and drought, and improve overall water security and livelihoods for rural populations. Additionally, rainwater harvesting contributes to environmental conservation by replenishing groundwater, reducing soil erosion, and supporting biodiversity in local ecosystems.

	\subsubsection{Scope}
	\begin{enumerate}[label=\textbf{\arabic*.}]
		\item \textbf{Environmental Impact} \\
		Rainwater harvesting helps in conserving water resources, reducing dependence on groundwater, and mitigating the effects of drought and water scarcity.
		\item \textbf{Water Security} \\
		It enhances water security by providing an alternative and sustainable source of water for various purposes such as drinking, sanitation, irrigation, and industrial use.
		\item \textbf{Community Development} \\
		Rainwater harvesting projects can contribute to community development by improving access to clean water, enhancing agricultural productivity, and supporting livelihoods, particularly in rural areas.
		\item \textbf{Climate Resilience} \\
		It enhances climate resilience by capturing and storing rainwater during wet periods for use during dry spells, thus helping communities adapt to changing weather patterns and extremes.
	\end{enumerate} \nborder \newpage

	\subsubsection{Dimensions}
	\begin{enumerate}[label=\textbf{\arabic*.}]
		\item \textbf{Technical} \\
		The technical dimension involves designing and implementing rainwater harvesting systems suitable for different contexts, including rooftop harvesting, surface water harvesting, and groundwater recharge.
		\item \textbf{Social} \\
		Social dimensions encompass community participation, capacity building, and awareness-raising activities to promote the adoption and sustainability of rainwater harvesting initiatives.
		\item \textbf{Legal and Policy} \\
		This dimension involves ensuring that rainwater harvesting projects comply with relevant laws, regulations, and policies governing water use and management at local, regional, and national levels.
		\item \textbf{Economic} \\
		Economic dimensions include the cost-effectiveness of rainwater harvesting compared to alternative water sources, as well as potential economic benefits such as increased agricultural productivity, reduced water bills, and job creation.
	\end{enumerate}

	\subsubsection{Cost}
	\begin{enumerate}[label=\textbf{\arabic*.}]
		\item \textbf{Initial Investment} \\
		The cost of rainwater harvesting systems varies depending on factors such as the size and complexity of the system, the type of harvesting technology used, and local labor and material costs. Initial investments may include the cost of infrastructure (e.g., tanks, pipes, filters), labor, and engineering/design services.

		\item \textbf{Operation and Maintenance} \\
		Ongoing costs include maintenance activities such as cleaning filters, repairing leaks, and replacing worn-out components. These costs are typically lower than initial investment costs but should be budgeted for to ensure the long-term functionality and sustainability of the system.
		\nborder \newpage
		\item \textbf{Cost-Benefit Analysis} \\
		Conducting a cost-benefit analysis can help assess the financial viability of rainwater harvesting projects by comparing the upfront costs with the potential savings or benefits achieved over time, such as reduced water bills, increased crop yields, and improved water security.
	\end{enumerate}

	\subsubsection{Beneficiaries}
	\begin{enumerate}[label=\textbf{\arabic*.}]
		\item \textbf{Households} \\
		Rainwater harvesting provides households with an alternative and sustainable source of water for various domestic purposes, including drinking, cooking, bathing, and sanitation.

		\item \textbf{Communities} \\
		Entire communities benefit from rainwater harvesting by improving access to clean water, particularly in rural areas where water scarcity is common. Community-level rainwater harvesting systems can serve multiple households and support collective water security.

		\item \textbf{Farmers} \\
		Rainwater harvesting enhances agricultural productivity by providing water for irrigation during dry periods, reducing reliance on erratic rainfall and groundwater pumping. Farmers can use harvested rainwater to sustain crops, improve yields, and diversify agricultural activities.

		\item \textbf{Ecosystems} \\
		Rainwater harvesting promotes environmental conservation by replenishing groundwater, reducing soil erosion, and supporting local ecosystems. Recharging aquifers and maintaining natural water cycles are critical for sustaining biodiversity and ecosystem services.
	\end{enumerate}

	\par Overall, rainwater harvesting has the potential to benefit a wide range of stakeholders, from individual households to entire ecosystems, by improving water security, enhancing livelihoods, and promoting environmental sustainability. \nborder \newpage

	\subsubsection{Implementation of the Scheme}
	\par Implementing rainwater harvesting involves several steps and considerations to ensure its effectiveness and sustainability. Here's an overview of the implementation process:

	\begin{enumerate}[label=\textbf{\arabic*.}]
		\item \textbf{Assessment and Planning}
		\begin{itemize}
			\item \textbf{Site Assessment} \\
			Evaluate the geographical, hydrological, and climatic conditions of the area to determine the suitability of rainwater harvesting and identify potential sources of rainwater runoff.
			\item \textbf{Water Demand Analysis} \\
			Assess the water needs and usage patterns of the target beneficiaries (e.g., households, communities, farms) to determine the scale and capacity of the rainwater harvesting system required.
			\item \textbf{Regulatory and Permitting} \\
			Research and comply with local regulations, permits, and zoning requirements related to rainwater harvesting, including water rights, building codes, and environmental regulations.
		\end{itemize}

		\item \textbf{Design and Engineering}
		\begin{itemize}
			\item \textbf{System Design} \\
			Develop a customized rainwater harvesting system design based on the site assessment, water demand analysis, and regulatory requirements. Consider factors such as catchment area, rainfall intensity, storage capacity, and treatment needs.
			\item \textbf{Component Selection} \\
			Select appropriate components for the rainwater harvesting system, including roof catchment surfaces, gutters, downspouts, filtration systems, storage tanks, distribution pipes, and overflow controls.
			\nborder \newpage
			\item \textbf{Engineering Specifications} \\
			Prepare detailed engineering specifications, drawings, and plans for the construction and installation of the rainwater harvesting system, ensuring compliance with technical standards and safety regulations.
		\end{itemize}

		\item \textbf{Construction and Installation}
		\begin{itemize}
			\item \textbf{Preparation} \\
			Clear the site and prepare the foundation for the rainwater harvesting infrastructure, ensuring proper alignment and stability of components.
			\item \textbf{Installation} \\
			Install the various components of the rainwater harvesting system according to the design specifications, including mounting gutters and downspouts, connecting pipes, positioning storage tanks, and integrating filtration and treatment systems.
			\item \textbf{Quality Control} \\
			Conduct quality control checks throughout the construction process to ensure the integrity, functionality, and durability of the rainwater harvesting system, addressing any defects or deviations promptly.
		\end{itemize}

		\item \textbf{Operation and Maintenance}
		\begin{itemize}
			\item \textbf{Training and Capacity Building} \\
			Provide training and capacity building to end-users and stakeholders on the operation, maintenance, and management of the rainwater harvesting system, including monitoring water quality, inspecting components, and troubleshooting common issues.
			\item \textbf{Regular Maintenance} \\
			Establish a routine maintenance schedule for inspecting, cleaning, and servicing the rainwater harvesting system to ensure optimal performance and longevity. Tasks may include cleaning gutters, checking for leaks, flushing tanks, and replacing filters or damaged components.
			\nborder \newpage
			\item \textbf{Record Keeping} \\
			Maintain accurate records of maintenance activities, water usage, and system performance to track effectiveness, identify trends, and inform future decision-making.
		\end{itemize}

		\item \textbf{Monitoring and Evaluation}
		\begin{itemize}
			\item \textbf{Performance Monitoring} \\
			Monitor the performance of the rainwater harvesting system regularly by measuring rainfall, water levels in storage tanks, and water quality parameters (e.g., turbidity, pH) to assess efficiency and effectiveness.
			\item \textbf{Feedback and Adaptation} \\
			Gather feedback from users and stakeholders to identify challenges, opportunities, and areas for improvement in the rainwater harvesting system. Use monitoring data to make informed decisions and adapt management practices as needed to optimize performance and address evolving needs.
		\end{itemize}

		\item \textbf{Community Engagement and Awareness}
		\begin{itemize}
			\item \textbf{Education and Outreach} \\
			Engage with the community through workshops, training sessions, and awareness campaigns to promote understanding of rainwater harvesting benefits, encourage participation, and foster a sense of ownership and responsibility among users.
			\item \textbf{Stakeholder Collaboration} \\
			Collaborate with local organizations, government agencies, NGOs, and other stakeholders to mobilize resources, build partnerships, and leverage expertise for successful implementation and sustainability of rainwater harvesting initiatives.
		\end{itemize}
	\end{enumerate}

	\par By following a systematic approach to implementation, rainwater harvesting projects can effectively harness rainwater resources, improve water security, and enhance resilience in communities while contributing to environmental sustainability and socio-economic development.

	\nborder \newpage

	\chapter{Module 5 $-$ FOOD WALK}
	\subsection{Regional foods of Karnataka}
	\subsubsection{Mundakki Upkari}
	\par Mundakki upkari is a simple dish that takes just 10 minutes to make. In the coastal areas of Kundapur, Udupi, and Mangalore, mundakki upkari is cooked in this manner.

	\begin{center}
		\img{.5}{Food/mundakki_upkari.jpg} \\ Mundakki Upkari
	\end{center}

	\noindent \textbf{Ingredients}
	\begin{itemize}
		\item 2 cups rice puffed
		\item 1 large tomato, coarsely chopped
		\item 1 onion, finely chopped
		\item 1 tblsp. red pepper flakes
		\item 1 tbsp. rasam powder
		\item 2 tablespoons fresh corriander, coarsely chopped season with salt to taste
		\nborder \newpage
		\item 4 tblsp. coconut oil
		\item 1 medium-sized peeled and shredded carrot
		\item 2 tablespoons raw mango grated (optional)
		\item 1 lemon
	\end{itemize}

	\noindent \vspace{1em} \textbf{Time to prepare : 10 minutes} \\ ~ \\
	\textbf{Method of Preparation}
	\begin{itemize}
		\item Puffed rice, salt, red chili powder, and rasam powder should all be combined in a mixing dish.
		\item Mix in the coconut oil until the powders are evenly distributed throughout the puffed rice.
		\item On a dish, place this spicy puffed rice. Mix well with finely chopped tomato, finely chopped onions, peeled, grated carrot, finely chopped fresh coriander, and lemon juice.
	\end{itemize} \nborder \newpage

	\nborder
	\subsubsection{Pankam}
	\par Panakam is a Sanskrit word that translates to a “sweet drink”. In the simplest and the most traditional form it is simply water sweetened with jaggery, flavoured with crushed cardamoms and spiced with ground black pepper. This basic drink is known as Panakam in Telugu, Panaka in Kannada and Panagam in Tamil.

	\begin{center}
		\img{.25}{Food/panakam.jpg} \\ Panakam
	\end{center}

	\noindent \textbf{Ingredients}
	\begin{itemize}
		\item 4 tbsp. Jaggery (Grated)
		\item 2 cups Water
		\item Pinch cardamom powder (1 large cardamom)
		\item Pinch dry ginger powder
		\item Pinch pepper (coarsely crushed)
	\end{itemize}

	\noindent \vspace{1em} \textbf{Time to prepare : 1 minute} \\ ~ \\
	\textbf{Method of Preparation}
	\begin{enumerate}[label=\textbf{\arabic*.}]
		\item  Mix jaggery and water and stir till jaggery melts. Strain to remove impurities.
		\item  Add cardamom, ginger powder and crushed pepper.
	\end{enumerate}
\end{document}